\documentclass[12pt]{article}

\usepackage{epsfig}
\usepackage{graphics}
\usepackage{color}
\usepackage{amsmath}

\begin{document}

\title{Solution: Sonny Bayes Robs a Train} 
\author{David Johnston \\ThoughtWorks\\dajohnst@thoughtworks.com}
\maketitle

\section{A Bayesian Approach}

I know of no way of solving this problem without using a Bayesian approach. Some people will
get the problem right without knowing anything about Bayes theorem but that's because people
usually think in a Bayesian way naturally. 

This problem demonstrates most starkly the need to introduce a reasonable prior distribution. 
Let's introduce the three variables that define the problem. Let $x$, denote the unknown amount of gold initially 
loaded into each of the trains. We'll make it an integer in dollars just so we can avoid integrals. 
Therefore one of the running trains has $x$ amount of gold and the other has $2x$. 
Let us also define another parameter $L$, a Boolean defining their Luck. $L$ is True if their random guess resulted
in them arriving first at the train with more gold and False otherwise. Finally, there is $D$, the estimated
amount of gold they will find in the first train. $L$ and $x$ are independent variables. There is no reason that
their random guess at a train will be affected by the amount of gold put into the trains. 
However they are \emph{conditionally} dependent on $D$ by the knowledge that one trains has twice as
much gold as the other. 

\section{The General Solution}

We will proceed by writing down the probability that we desire and apply Bayes theorem to get someting that
we can calculate numerically. We are interested in the posterior probability of $L$ given the new data. This is 
actually a marginal distribution; marginalized over x.
\begin{equation}
P(L | D) = \sum_x P(L, x | D)
\end{equation}
Now we can apply Bayes theorem to $P(L,x | D)$
\begin{eqnarray}
P(L, x | D) &=& P(D | L, x) P(L,x) / P(D)\\
&\propto& P(D | L, x) P(L) P(x)
\end{eqnarray}
and so 
\begin{eqnarray}
P(L | D) & \propto &\sum_x P(D | L, x) P(L) P(x) \\
P(L | D) & \propto &\sum_x P(D | L, x) P(x)
\end{eqnarray}

As is common, we don't care about the normalization constant $P(D)$ so we throw it away and change it from an equation to 
a proprtionality statement. If that is confusing, you can keep it. You will see that it will drop out anyway. 
We use the independence of $L$ and $x$ in saying $P(L,x) = P(L) P(x)$. We chose the first train at random which implies $P(L) =0.5$ for both values 
of $L$. It's just another constant factor and so we dropped it from the above equation. 

This is now in a form that we can work with. That is, we can logically assign numbers to each of these three terms 
in the sum from the information given in the story. 

Let's start with $P(D | L,x)$. This is called the likelihood of the data. What is the value of this term? It's usualy
helpful to spell these things out in words. This is: What is the probability of observing D=10 (or any amount) 
given that we know $L$ and also know $x$. For example, if $x=5$ million and $L=True$, the probability is 1 
because we were lucky to choose the best train and 2 times $x=5$ is 10. For any other value of $x$ it will be zero. 
If instead we know $L=False$, this will be zero for every value except when $x=10$. 

Finally, there is $P(x)$. This is the prior distrubution on $x$. It embodies our expectation of the value of $x$ before 
receiving the new data that there is 10 million worth of gold on one of the trains. Often, it is the case that the 
prior is not \emph{that} important and, in fact, it gets less important as you add more data. But in this problem there is 
no more data to be had and so the prior determines the problem entirely. 

Let's get back to our equation and calculate $P(L | D)$ for each of the possible values of $L$ which are True and False.
As stated $P(D| L,x)$ is 1 for only one value, depending on $L$ and zero for every other and so our sum collapse down 
to a single term. 

$P(L|D) \propto
\begin{cases}

  P(x=D/2), & \text{if } L = True, \\
  P(x=D), & \text{otherwise}.
\end{cases}$
 
Now we know (assuming we trust our informant) that $P(L= True | D) + P(L= False | D) =1$ because, 
regardless of the data, we know that $L$ is either True or False and so the properly normalized $P(L | D)$, 
for either $L$, is clearly. 

\begin{equation}
P(L | D) = \frac{P(L | D)}{P(L=True | D) + P(L=False | D)}
\end{equation}

So for example, 
\begin{equation}
P(L=True | D) = \frac{1}{1+\frac{P(L=False | D)}{P(L=True|D)}}
\end{equation}
The last step in the formal solution is to plug in the value of this ratio. Again notice how any
common constant factors would have dropped out in this ratio.  
\begin{equation}
P(L=True | D) = \frac{1}{1+\frac{P(x=D)}{P(x=D/2)}}
\label{eq:final}
\end{equation}
As mentioned before, the answer of whether $P(L=True | D)$ (we should take the gold) is greater than
$P(L=False | D)$ (we should rob the other train) depends entirely on our prior distribution evaluated at 
our one data point, $D=10$, and half that value. 

\section{Apply the Prior}
By now, we see that everything depends on our prior. But what should we use for that? If we had no information
at all, we would be in a pinch. Using a uniform prior would result in $P(L=False | D)=0.5$ regardless of the data. 
Fortunately we do have prior information as we have seen the amount of gold on three other attempts. Even so, we
have to make some assumptions about the smoothness of the prior or adapt some functional form. For simplicity, we
can start with the normal distribution, $P(x) \propto \exp(-0.5 \left[(x-\mu)/\sigma \right]^2)$ (though log-normal 
might make more sense). From the sample, we can see that the mean $\mu$ is about 11.1 and the standard deviation 
$\sigma$ is about 3. Using a little algebra, it is easy to show 
\begin{equation}
P(L=True | D) = \frac{1}{1+ \exp \left(D [\mu- 3 D/4]/(2 \sigma^2)\right)}
\label{eq:crit}
\end{equation}

Finally, we can see what the true professional would do. All that matters is whether $P(L=True | D)$ is less than 0.5.
If it is, they are more likely to get more gold by robbing the other train. If not, they take the gold on the first train. 
Note that this is completely determined by the sign of $[\mu- 3 D/4]$ in the expression above. If $D < 4/3 \mu$, they are
better off switching to the other train. Since they know that $\mu=11.1$ before they even pick a train, they should know
that all they need to do is determine whether the first train has more than 11.1*4/3=14.8 million in gold. As soon as they 
determine that, they can decide what to do. 

In this case $D=10 < 14.8$ so it is better to rob the other train. What's the probability of success? Now that 
involves $\sigma$ as well. Plugging this in to the above with $\sigma=3$ gives us $P(L=True | D) = 0.12$. 
Thus, they have an $88\%$ chance of success if they switch. 


 



 


\end{document}
